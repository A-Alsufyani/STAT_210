% Options for packages loaded elsewhere
\PassOptionsToPackage{unicode}{hyperref}
\PassOptionsToPackage{hyphens}{url}
\documentclass[
]{article}
\usepackage{xcolor}
\usepackage[margin=1in]{geometry}
\usepackage{amsmath,amssymb}
\setcounter{secnumdepth}{-\maxdimen} % remove section numbering
\usepackage{iftex}
\ifPDFTeX
  \usepackage[T1]{fontenc}
  \usepackage[utf8]{inputenc}
  \usepackage{textcomp} % provide euro and other symbols
\else % if luatex or xetex
  \usepackage{unicode-math} % this also loads fontspec
  \defaultfontfeatures{Scale=MatchLowercase}
  \defaultfontfeatures[\rmfamily]{Ligatures=TeX,Scale=1}
\fi
\usepackage{lmodern}
\ifPDFTeX\else
  % xetex/luatex font selection
\fi
% Use upquote if available, for straight quotes in verbatim environments
\IfFileExists{upquote.sty}{\usepackage{upquote}}{}
\IfFileExists{microtype.sty}{% use microtype if available
  \usepackage[]{microtype}
  \UseMicrotypeSet[protrusion]{basicmath} % disable protrusion for tt fonts
}{}
\makeatletter
\@ifundefined{KOMAClassName}{% if non-KOMA class
  \IfFileExists{parskip.sty}{%
    \usepackage{parskip}
  }{% else
    \setlength{\parindent}{0pt}
    \setlength{\parskip}{6pt plus 2pt minus 1pt}}
}{% if KOMA class
  \KOMAoptions{parskip=half}}
\makeatother
\usepackage{color}
\usepackage{fancyvrb}
\newcommand{\VerbBar}{|}
\newcommand{\VERB}{\Verb[commandchars=\\\{\}]}
\DefineVerbatimEnvironment{Highlighting}{Verbatim}{commandchars=\\\{\}}
% Add ',fontsize=\small' for more characters per line
\usepackage{framed}
\definecolor{shadecolor}{RGB}{248,248,248}
\newenvironment{Shaded}{\begin{snugshade}}{\end{snugshade}}
\newcommand{\AlertTok}[1]{\textcolor[rgb]{0.94,0.16,0.16}{#1}}
\newcommand{\AnnotationTok}[1]{\textcolor[rgb]{0.56,0.35,0.01}{\textbf{\textit{#1}}}}
\newcommand{\AttributeTok}[1]{\textcolor[rgb]{0.13,0.29,0.53}{#1}}
\newcommand{\BaseNTok}[1]{\textcolor[rgb]{0.00,0.00,0.81}{#1}}
\newcommand{\BuiltInTok}[1]{#1}
\newcommand{\CharTok}[1]{\textcolor[rgb]{0.31,0.60,0.02}{#1}}
\newcommand{\CommentTok}[1]{\textcolor[rgb]{0.56,0.35,0.01}{\textit{#1}}}
\newcommand{\CommentVarTok}[1]{\textcolor[rgb]{0.56,0.35,0.01}{\textbf{\textit{#1}}}}
\newcommand{\ConstantTok}[1]{\textcolor[rgb]{0.56,0.35,0.01}{#1}}
\newcommand{\ControlFlowTok}[1]{\textcolor[rgb]{0.13,0.29,0.53}{\textbf{#1}}}
\newcommand{\DataTypeTok}[1]{\textcolor[rgb]{0.13,0.29,0.53}{#1}}
\newcommand{\DecValTok}[1]{\textcolor[rgb]{0.00,0.00,0.81}{#1}}
\newcommand{\DocumentationTok}[1]{\textcolor[rgb]{0.56,0.35,0.01}{\textbf{\textit{#1}}}}
\newcommand{\ErrorTok}[1]{\textcolor[rgb]{0.64,0.00,0.00}{\textbf{#1}}}
\newcommand{\ExtensionTok}[1]{#1}
\newcommand{\FloatTok}[1]{\textcolor[rgb]{0.00,0.00,0.81}{#1}}
\newcommand{\FunctionTok}[1]{\textcolor[rgb]{0.13,0.29,0.53}{\textbf{#1}}}
\newcommand{\ImportTok}[1]{#1}
\newcommand{\InformationTok}[1]{\textcolor[rgb]{0.56,0.35,0.01}{\textbf{\textit{#1}}}}
\newcommand{\KeywordTok}[1]{\textcolor[rgb]{0.13,0.29,0.53}{\textbf{#1}}}
\newcommand{\NormalTok}[1]{#1}
\newcommand{\OperatorTok}[1]{\textcolor[rgb]{0.81,0.36,0.00}{\textbf{#1}}}
\newcommand{\OtherTok}[1]{\textcolor[rgb]{0.56,0.35,0.01}{#1}}
\newcommand{\PreprocessorTok}[1]{\textcolor[rgb]{0.56,0.35,0.01}{\textit{#1}}}
\newcommand{\RegionMarkerTok}[1]{#1}
\newcommand{\SpecialCharTok}[1]{\textcolor[rgb]{0.81,0.36,0.00}{\textbf{#1}}}
\newcommand{\SpecialStringTok}[1]{\textcolor[rgb]{0.31,0.60,0.02}{#1}}
\newcommand{\StringTok}[1]{\textcolor[rgb]{0.31,0.60,0.02}{#1}}
\newcommand{\VariableTok}[1]{\textcolor[rgb]{0.00,0.00,0.00}{#1}}
\newcommand{\VerbatimStringTok}[1]{\textcolor[rgb]{0.31,0.60,0.02}{#1}}
\newcommand{\WarningTok}[1]{\textcolor[rgb]{0.56,0.35,0.01}{\textbf{\textit{#1}}}}
\usepackage{graphicx}
\makeatletter
\newsavebox\pandoc@box
\newcommand*\pandocbounded[1]{% scales image to fit in text height/width
  \sbox\pandoc@box{#1}%
  \Gscale@div\@tempa{\textheight}{\dimexpr\ht\pandoc@box+\dp\pandoc@box\relax}%
  \Gscale@div\@tempb{\linewidth}{\wd\pandoc@box}%
  \ifdim\@tempb\p@<\@tempa\p@\let\@tempa\@tempb\fi% select the smaller of both
  \ifdim\@tempa\p@<\p@\scalebox{\@tempa}{\usebox\pandoc@box}%
  \else\usebox{\pandoc@box}%
  \fi%
}
% Set default figure placement to htbp
\def\fps@figure{htbp}
\makeatother
\setlength{\emergencystretch}{3em} % prevent overfull lines
\providecommand{\tightlist}{%
  \setlength{\itemsep}{0pt}\setlength{\parskip}{0pt}}
\usepackage{bookmark}
\IfFileExists{xurl.sty}{\usepackage{xurl}}{} % add URL line breaks if available
\urlstyle{same}
\hypersetup{
  hidelinks,
  pdfcreator={LaTeX via pandoc}}

\title{STAT 210\\
Applied Statistics and Data Analysis\\
First Exam}
\author{}
\date{\vspace{-2.5em}October 25, 2025}

\begin{document}
\maketitle

\begin{center}\color{red}
{\Large \textbf{You are not allowed to use AI tools to solve this exam}}
\end{center}

\textbf{You are reminded to adhere to the academic integrity code
established at KAUST.}

\textbf{This exam is open notes and open book but not open internet. You
are not allowed to use the internet except for downloading the exam and
uploading the solution.}

\textbf{Show complete solutions to get full credit. Writing code is not
enough to answer a question. Your comments are more important than the
code. Do not write comments inside chunks. Label your graphs
appropriately. Please identify the files you submit with your surname}

\bigskip
\begin{center}\color{red}
{\textbf{For all tests in this exam use a significance level of $\mathbf \alpha$ = 0.01 \\unless otherwise specified}}
\end{center}

\subsection{Question 1 (50 points)}\label{question-1-50-points}

An industrial engineer is testing whether a new and cheaper machine
calibration method (Method B) preserves production performance compared
to the current method (Method A). The main quality measure is the
tensile strength of metal rods (in MPa) produced by the machines.

\begin{enumerate}
\def\labelenumi{\alph{enumi})}
\tightlist
\item
  Historically, rods produced using Method A have an average tensile
  strength of 150 MPa. After implementing Method B, a random sample of
  18 rods is tested, yielding the following strengths:
\end{enumerate}

\begin{Shaded}
\begin{Highlighting}[]
\NormalTok{methodB }\OtherTok{\textless{}{-}} \FunctionTok{c}\NormalTok{(}\FloatTok{152.4}\NormalTok{, }\FloatTok{155.2}\NormalTok{, }\FloatTok{149.6}\NormalTok{, }\FloatTok{153.8}\NormalTok{, }\FloatTok{156.1}\NormalTok{, }\FloatTok{150.9}\NormalTok{, }\FloatTok{154.5}\NormalTok{, }\FloatTok{157.0}\NormalTok{, }\FloatTok{151.8}\NormalTok{,}
             \FloatTok{155.6}\NormalTok{, }\FloatTok{150.3}\NormalTok{, }\FloatTok{153.0}\NormalTok{, }\FloatTok{154.2}\NormalTok{, }\FloatTok{156.7}\NormalTok{, }\FloatTok{149.8}\NormalTok{, }\FloatTok{152.9}\NormalTok{, }\FloatTok{155.1}\NormalTok{, }\FloatTok{154.0}\NormalTok{)}
\end{Highlighting}
\end{Shaded}

What parametric test would you use to compare the new calibration method
with the reference value? State clearly what hypotheses you are testing
and which assumptions are needed for the test. Explain why you think
they are satisfied. Describe the test statistic and calculate its value.
Describe the sampling distribution and explicitly identify the errors of
types I and II. Carry out this test and discuss the results.

\begin{enumerate}
\def\labelenumi{\alph{enumi})}
\setcounter{enumi}{1}
\tightlist
\item
  To compare the two methods directly, two machines are run for one day
  each: Machine A uses the standard calibration, Machine B uses the new
  calibration method. The tensile strengths (in MPa) of 12 rods from
  each machine are recorded:
\end{enumerate}

\begin{Shaded}
\begin{Highlighting}[]
\NormalTok{machineA }\OtherTok{\textless{}{-}} \FunctionTok{c}\NormalTok{(}\FloatTok{149.2}\NormalTok{, }\FloatTok{151.0}\NormalTok{, }\FloatTok{148.7}\NormalTok{, }\FloatTok{150.1}\NormalTok{, }\FloatTok{149.8}\NormalTok{, }\FloatTok{152.3}\NormalTok{, }\FloatTok{150.5}\NormalTok{, }\FloatTok{151.2}\NormalTok{, }
              \FloatTok{148.9}\NormalTok{, }\FloatTok{149.7}\NormalTok{, }\FloatTok{150.9}\NormalTok{, }\FloatTok{149.4}\NormalTok{)}
\NormalTok{machineB }\OtherTok{\textless{}{-}} \FunctionTok{c}\NormalTok{(}\FloatTok{153.4}\NormalTok{, }\FloatTok{155.1}\NormalTok{, }\FloatTok{152.8}\NormalTok{, }\FloatTok{154.3}\NormalTok{, }\FloatTok{156.0}\NormalTok{, }\FloatTok{155.6}\NormalTok{, }\FloatTok{153.9}\NormalTok{, }\FloatTok{154.8}\NormalTok{, }
              \FloatTok{152.5}\NormalTok{, }\FloatTok{155.2}\NormalTok{, }\FloatTok{154.0}\NormalTok{, }\FloatTok{153.7}\NormalTok{)}
\end{Highlighting}
\end{Shaded}

The engineer now wants to specifically test whether Machine B produces
stronger rods than Machine A. What parametric test would be adequate for
comparing the tensile strengths corresponding to the two methods? State
clearly what hypothesis you are testing and which assumptions are needed
for the test. Explain why you think they are satisfied. Carry out this
test and discuss the results.

\begin{enumerate}
\def\labelenumi{\alph{enumi})}
\setcounter{enumi}{2}
\tightlist
\item
  In a controlled laboratory experiment, 10 batches of raw material were
  processed under both calibration methods (A and B). The same batch of
  material was used for both methods to eliminate raw material
  variation. The measured tensile strength results (MPa) are:
\end{enumerate}

\begin{Shaded}
\begin{Highlighting}[]
\NormalTok{mtdA }\OtherTok{\textless{}{-}} \FunctionTok{c}\NormalTok{(}\FloatTok{148.5}\NormalTok{, }\FloatTok{149.7}\NormalTok{, }\FloatTok{150.3}\NormalTok{, }\FloatTok{151.1}\NormalTok{, }\FloatTok{149.0}\NormalTok{, }\FloatTok{150.8}\NormalTok{, }\FloatTok{149.5}\NormalTok{, }\FloatTok{150.0}\NormalTok{, }\FloatTok{151.2}\NormalTok{, }\FloatTok{149.8}\NormalTok{)}
\NormalTok{mtdB }\OtherTok{\textless{}{-}} \FunctionTok{c}\NormalTok{(}\FloatTok{149.6}\NormalTok{, }\FloatTok{150.5}\NormalTok{, }\FloatTok{151.1}\NormalTok{, }\FloatTok{152.0}\NormalTok{, }\FloatTok{149.2}\NormalTok{, }\FloatTok{151.6}\NormalTok{, }\FloatTok{150.1}\NormalTok{, }\FloatTok{151.0}\NormalTok{, }\FloatTok{152.3}\NormalTok{, }\FloatTok{150.4}\NormalTok{)}
\end{Highlighting}
\end{Shaded}

The engineer wants to determine if Method B (\texttt{mtdB}) produces
stronger rods than Method A (\texttt{mtdA}). What parametric test would
you use in this case? State clearly what hypothesis you are testing and
which assumptions are needed for the test, and explain why you think
they are satisfied. Identify the type I and type II errors. Carry out
this test and discuss the results.

\begin{enumerate}
\def\labelenumi{(\alph{enumi})}
\setcounter{enumi}{3}
\tightlist
\item
  What non-parametric tests will be adequate for the problems in (a),
  (b), and (c)? What assumptions are needed, and why do you think they
  are satisfied? Perform these tests, discuss the results, and compare
  them with your previous results.
\end{enumerate}

\begin{center}\rule{0.5\linewidth}{0.5pt}\end{center}

\subsection{Question 2 (50 points)}\label{question-2-50-points}

An industrial engineer wants to compare five different lubricants used
in a manufacturing process to determine their effect on the mean
friction coefficient of machine parts. The lubricants are coded
\texttt{A,\ B,\ C,\ D}, and \texttt{E}. Each lubricant is tested on five
identical machines under the same operating conditions, and the friction
coefficient is measured (lower values are better). The data is stored in
the file \texttt{XM125F\_Q2.csv}.

Do a complete analysis of variance model including the following: Plot
the data. Determine whether the lubricants have an effect on the mean
friction coefficient through a hypothesis test and state explicitly the
null and alternative hypotheses. Obtain the estimated values for the
cell (marginal) means and the effects. Write the equation for the model
and state explicitly the assumptions on which the model is based. Plot
the diagnostic charts and comment on them. Use Levene's and
Shapiro-Wilk's tests also. Use Tukey's HSD procedure to make pairwise
comparisons and comment on the results. Use a non-parametric alternative
to the analysis of variance and compare the results. Give your comments
on every step that you take. If the objective is a lower friction
coefficient, which design would you select and why?

\begin{Shaded}
\begin{Highlighting}[]
\FunctionTok{library}\NormalTok{(car)}
\end{Highlighting}
\end{Shaded}

\begin{verbatim}
## Loading required package: carData
\end{verbatim}

\begin{Shaded}
\begin{Highlighting}[]
\NormalTok{data }\OtherTok{\textless{}{-}} \FunctionTok{read.csv}\NormalTok{(}\StringTok{"XM125F\_Q2.csv"}\NormalTok{, }\AttributeTok{header =} \ConstantTok{TRUE}\NormalTok{) }\CommentTok{\# read the file}
\FunctionTok{str}\NormalTok{(data)}
\end{Highlighting}
\end{Shaded}

\begin{verbatim}
## 'data.frame':    25 obs. of  2 variables:
##  $ friction : num  0.41 0.39 0.37 0.36 0.4 0.35 0.33 0.34 0.36 0.35 ...
##  $ lubricant: chr  "A" "A" "A" "A" ...
\end{verbatim}

\begin{Shaded}
\begin{Highlighting}[]
\NormalTok{data}\SpecialCharTok{$}\NormalTok{lubricant }\OtherTok{\textless{}{-}} \FunctionTok{as.factor}\NormalTok{(data}\SpecialCharTok{$}\NormalTok{lubricant) }\CommentTok{\#make the lubricant a factor instead of character}
\FunctionTok{str}\NormalTok{(data)}
\end{Highlighting}
\end{Shaded}

\begin{verbatim}
## 'data.frame':    25 obs. of  2 variables:
##  $ friction : num  0.41 0.39 0.37 0.36 0.4 0.35 0.33 0.34 0.36 0.35 ...
##  $ lubricant: Factor w/ 5 levels "A","B","C","D",..: 1 1 1 1 1 2 2 2 2 2 ...
\end{verbatim}

\begin{Shaded}
\begin{Highlighting}[]
\FunctionTok{boxplot}\NormalTok{(friction }\SpecialCharTok{\textasciitilde{}}\NormalTok{ lubricant, }\AttributeTok{data =}\NormalTok{ data) }\CommentTok{\#boxplot of the friction for each lubricant}
\end{Highlighting}
\end{Shaded}

\pandocbounded{\includegraphics[keepaspectratio]{Exam1-2025F-1-_files/figure-latex/unnamed-chunk-5-1.pdf}}

\begin{Shaded}
\begin{Highlighting}[]
\CommentTok{\#points(friction \textasciitilde{} lubricant,data = data, col = "purple", pch = 16, ) \#plot the points on the boxplot}
\end{Highlighting}
\end{Shaded}

For the parametric test, we will use an anova with the hypotheses as
follows: H0: mu\_A = mu\_B = mu\_C = mu\_D = mu\_E H1: at least one mean
is different.

H0 basically means that all lubricant types are the same and have no
effect.

The anova model assumptions are independence, normality, and equal
variance.

\begin{Shaded}
\begin{Highlighting}[]
\NormalTok{mod1}\OtherTok{\textless{}{-}} \FunctionTok{aov}\NormalTok{(friction }\SpecialCharTok{\textasciitilde{}}\NormalTok{ lubricant, }\AttributeTok{data =}\NormalTok{ data)}
\FunctionTok{summary}\NormalTok{(mod1)}
\end{Highlighting}
\end{Shaded}

\begin{verbatim}
##             Df Sum Sq Mean Sq F value  Pr(>F)    
## lubricant    4 0.0374 0.00935    43.7 1.3e-09 ***
## Residuals   20 0.0043 0.00021                    
## ---
## Signif. codes:  0 '***' 0.001 '**' 0.01 '*' 0.05 '.' 0.1 ' ' 1
\end{verbatim}

Since the p-value of the anova (1.3e-9) is lower than our signficance
level of 0.01 we reject the null hypothesis and we have evidence to say
that at least one lubricant is different.

we then find the cell means and standard error.

\begin{Shaded}
\begin{Highlighting}[]
\FunctionTok{model.tables}\NormalTok{(mod1, }\StringTok{\textquotesingle{}mean\textquotesingle{}}\NormalTok{, }\AttributeTok{se =}\NormalTok{ T)}
\end{Highlighting}
\end{Shaded}

\begin{verbatim}
## Tables of means
## Grand mean
##        
## 0.3404 
## 
##  lubricant 
## lubricant
##     A     B     C     D     E 
## 0.386 0.346 0.310 0.376 0.284 
## 
## Standard errors for differences of means
##         lubricant
##          0.009252
## replic.         5
\end{verbatim}

\begin{Shaded}
\begin{Highlighting}[]
\FunctionTok{model.tables}\NormalTok{(mod1, }\AttributeTok{se =}\NormalTok{ T)}
\end{Highlighting}
\end{Shaded}

\begin{verbatim}
## Tables of effects
## 
##  lubricant 
## lubricant
##       A       B       C       D       E 
##  0.0456  0.0056 -0.0304  0.0356 -0.0564 
## 
## Standard errors of effects
##         lubricant
##          0.006542
## replic.         5
\end{verbatim}

Equation of the anova model: Y\_ij = mu + tau\_i + epsilon\_ij

Where: Y\_ij = weight gain for the j-th observation in the i-th diet
group mu = overall mean weight gain tau\_i = effect of the i-th diet
(deviation from overall mean) epsilon\_ij = random error term i = 1, 2,
3, 4, 5 (diet groups: ctrl, dt1, dt2, dt3, dt4) j = 1, 2, \ldots, n\_i
(observations within each group)

we then print out the diagnostic plots:

\begin{Shaded}
\begin{Highlighting}[]
\FunctionTok{par}\NormalTok{(}\AttributeTok{mfrow=}\FunctionTok{c}\NormalTok{(}\DecValTok{2}\NormalTok{,}\DecValTok{2}\NormalTok{))}
\FunctionTok{plot}\NormalTok{(mod1)}
\end{Highlighting}
\end{Shaded}

\pandocbounded{\includegraphics[keepaspectratio]{Exam1-2025F-1-_files/figure-latex/unnamed-chunk-8-1.pdf}}

From the plot we see the following: -Residuals vs fitted: the line is
orizantal at or very near 0, which indicates linearity -QQ Residuals:
The data is aligned with the normal line, which indicates normality.
-Scale-Location: the line is mostly horizantal with some tails, further
testing may be needed to decide if variance is equal (levene test)
-Constant lev: plot shows no influential outliers

Testing for equal variance:

\begin{Shaded}
\begin{Highlighting}[]
\FunctionTok{leveneTest}\NormalTok{(mod1)}
\end{Highlighting}
\end{Shaded}

\begin{verbatim}
## Levene's Test for Homogeneity of Variance (center = median)
##       Df F value Pr(>F)
## group  4    0.78   0.55
##       20
\end{verbatim}

With a p-value of 0.551 \textgreater\textgreater{} 0.01, we fail to
reject the null hypothesis and evidence suggests that we have equal
variance.

Testing for normality:

\begin{Shaded}
\begin{Highlighting}[]
\FunctionTok{shapiro.test}\NormalTok{(}\FunctionTok{rstandard}\NormalTok{(mod1))}
\end{Highlighting}
\end{Shaded}

\begin{verbatim}
## 
##  Shapiro-Wilk normality test
## 
## data:  rstandard(mod1)
## W = 0.971, p-value = 0.66
\end{verbatim}

with a p-value of 0.6644 \textgreater\textgreater{} 0.01, we fali to
reject the null hypothesis and evidence suggests that data is normal

Next we use TukeyHSD with confidence level = 0.99 to further analyze the
data:

\begin{Shaded}
\begin{Highlighting}[]
\NormalTok{(mod1.tky }\OtherTok{\textless{}{-}} \FunctionTok{TukeyHSD}\NormalTok{(mod1, }\AttributeTok{conf.level =} \FloatTok{0.99}\NormalTok{))}
\end{Highlighting}
\end{Shaded}

\begin{verbatim}
##   Tukey multiple comparisons of means
##     99% family-wise confidence level
## 
## Fit: aov(formula = friction ~ lubricant, data = data)
## 
## $lubricant
##       diff        lwr        upr   p adj
## B-A -0.040 -0.0746294 -0.0053706 0.00272
## C-A -0.076 -0.1106294 -0.0413706 0.00000
## D-A -0.010 -0.0446294  0.0246294 0.81410
## E-A -0.102 -0.1366294 -0.0673706 0.00000
## C-B -0.036 -0.0706294 -0.0013706 0.00719
## D-B  0.030 -0.0046294  0.0646294 0.02968
## E-B -0.062 -0.0966294 -0.0273706 0.00001
## D-C  0.066  0.0313706  0.1006294 0.00001
## E-C -0.026 -0.0606294  0.0086294 0.07215
## E-D -0.092 -0.1266294 -0.0573706 0.00000
\end{verbatim}

\begin{Shaded}
\begin{Highlighting}[]
\FunctionTok{par}\NormalTok{(}\AttributeTok{mfrow =} \FunctionTok{c}\NormalTok{(}\DecValTok{1}\NormalTok{,}\DecValTok{1}\NormalTok{))}
\FunctionTok{plot}\NormalTok{(mod1.tky)}
\end{Highlighting}
\end{Shaded}

\pandocbounded{\includegraphics[keepaspectratio]{Exam1-2025F-1-_files/figure-latex/unnamed-chunk-11-1.pdf}}

from the printed data, we see that the following pairs have p values
\textgreater{} 0.01, therefore A and D are indistinguishable B and D are
indistinguishable C and E are indistinguishable

from the difference we see that C has lower friction than A, B, and D
and its indistinguishable from E.

therefore if we want lower friction we go with C or E, and for higher
friction we go for lubricants A, B, or D

For a non-parametric alternative to the anova we use the Kruskal-Wallis
rank sum test it has the same assumptions of the anova except for the
normality requirement.

\begin{Shaded}
\begin{Highlighting}[]
\FunctionTok{kruskal.test}\NormalTok{(friction }\SpecialCharTok{\textasciitilde{}}\NormalTok{ lubricant, }\AttributeTok{data =}\NormalTok{ data)}
\end{Highlighting}
\end{Shaded}

\begin{verbatim}
## 
##  Kruskal-Wallis rank sum test
## 
## data:  friction by lubricant
## Kruskal-Wallis chi-squared = 21.4, df = 4, p-value = 0.00026
\end{verbatim}

with a p-value of 0.0002605 \textless\textless{} 0.01, it leads to the
same conclusion as the anova model we did.

Since the objective is to get the lowest amount of friction. data
suggests that we can group the lubricants into two groups. A,B,D which
are indistinguishable and C, E which are also indistinguishable.

C,E showed signficant decrease in friction compared to the other group
of A,B,D thus choosing C or E is recommended for the wanted purpose

\end{document}

% Options for packages loaded elsewhere
\PassOptionsToPackage{unicode}{hyperref}
\PassOptionsToPackage{hyphens}{url}
%
\documentclass[
]{article}
\usepackage{amsmath,amssymb}
\usepackage{iftex}
\ifPDFTeX
  \usepackage[T1]{fontenc}
  \usepackage[utf8]{inputenc}
  \usepackage{textcomp} % provide euro and other symbols
\else % if luatex or xetex
  \usepackage{unicode-math} % this also loads fontspec
  \defaultfontfeatures{Scale=MatchLowercase}
  \defaultfontfeatures[\rmfamily]{Ligatures=TeX,Scale=1}
\fi
\usepackage{lmodern}
\ifPDFTeX\else
  % xetex/luatex font selection
\fi
% Use upquote if available, for straight quotes in verbatim environments
\IfFileExists{upquote.sty}{\usepackage{upquote}}{}
\IfFileExists{microtype.sty}{% use microtype if available
  \usepackage[]{microtype}
  \UseMicrotypeSet[protrusion]{basicmath} % disable protrusion for tt fonts
}{}
\makeatletter
\@ifundefined{KOMAClassName}{% if non-KOMA class
  \IfFileExists{parskip.sty}{%
    \usepackage{parskip}
  }{% else
    \setlength{\parindent}{0pt}
    \setlength{\parskip}{6pt plus 2pt minus 1pt}}
}{% if KOMA class
  \KOMAoptions{parskip=half}}
\makeatother
\usepackage{xcolor}
\usepackage[margin=1in]{geometry}
\usepackage{graphicx}
\makeatletter
\newsavebox\pandoc@box
\newcommand*\pandocbounded[1]{% scales image to fit in text height/width
  \sbox\pandoc@box{#1}%
  \Gscale@div\@tempa{\textheight}{\dimexpr\ht\pandoc@box+\dp\pandoc@box\relax}%
  \Gscale@div\@tempb{\linewidth}{\wd\pandoc@box}%
  \ifdim\@tempb\p@<\@tempa\p@\let\@tempa\@tempb\fi% select the smaller of both
  \ifdim\@tempa\p@<\p@\scalebox{\@tempa}{\usebox\pandoc@box}%
  \else\usebox{\pandoc@box}%
  \fi%
}
% Set default figure placement to htbp
\def\fps@figure{htbp}
\makeatother
\setlength{\emergencystretch}{3em} % prevent overfull lines
\providecommand{\tightlist}{%
  \setlength{\itemsep}{0pt}\setlength{\parskip}{0pt}}
\setcounter{secnumdepth}{-\maxdimen} % remove section numbering
\usepackage{bookmark}
\IfFileExists{xurl.sty}{\usepackage{xurl}}{} % add URL line breaks if available
\urlstyle{same}
\hypersetup{
  pdftitle={STAT 210 - Applied Statistics and Data Analysis},
  hidelinks,
  pdfcreator={LaTeX via pandoc}}

\title{STAT 210 - Applied Statistics and Data Analysis}
\usepackage{etoolbox}
\makeatletter
\providecommand{\subtitle}[1]{% add subtitle to \maketitle
  \apptocmd{\@title}{\par {\large #1 \par}}{}{}
}
\makeatother
\subtitle{Homework 6}
\author{}
\date{\vspace{-2.5em}Due on November 09, 2025}

\begin{document}
\maketitle

\subsection{Instructions}\label{instructions}

\begin{itemize}
\tightlist
\item
  You cannot use artificial intelligence tools to solve this homework.\\
\item
  Show complete solutions to get full credit. Writing code is not enough
  to answer a question.\\
\item
  Your comments are more important than the code. Do not write comments
  in chunks.\\
\item
  Label your graphs appropriately.\\
\item
  For all tests in this homework, use a significance level of
  \textbackslash( \alpha = 0.02 \textbackslash).
\end{itemize}

\begin{center}\rule{0.5\linewidth}{0.5pt}\end{center}

\subsection{Question 1 (50 pts)}\label{question-1-50-pts}

The data for this question is stored in the file \texttt{CHFLS} in the
library \texttt{HSAUR3} and comes from a survey of 60 villages and urban
neighborhoods in China published in 2003.\\
It has 1534 observations of 10 variables, but we will focus on
\texttt{R\_age} (age), \texttt{R\_happy} (self-reported happiness), and
\texttt{R\_region} (region).

\subsubsection{(a)}\label{a}

Create a new data frame called \texttt{df1} that only includes
\texttt{R\_age}, \texttt{R\_happy}, and \texttt{R\_region}.\\
Check whether the new data frame has missing data.\\
Explore the distribution of \texttt{R\_age} for the different regions.\\
Do boxplots of age as a function of region and comment on what you
observe.\\
Calculate mean, standard deviation, median, and interquartile range for
\texttt{R\_age} for each of the six regions and comment.

\end{document}
